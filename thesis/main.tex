\documentclass[14pt]{spbau-diploma}
\usepackage{listings}
\usepackage{verbatim}
\usepackage{amsmath}
\usepackage{booktabs}
\usepackage{multirow}
\usepackage{csquotes}
\usepackage{graphicx}
\usepackage[labelsep=period]{caption}
\usepackage{float}

\setotherlanguage{english}

\usepackage{soulutf8}
\usepackage{color}
\sethlcolor{yellow}

\newcommand{\citeisr}{\cite{isrpaper}}
\newcommand{\xvector}{\texttt{x-vector}}
\newcommand{\guesser}{\texttt{Guesser}}
\newcommand{\enquirer}{\texttt{Enquirer}}
\newcommand{\cbenquirer}{\texttt{Codebook\allowbreak{}Enquirer}}
\newcommand{\pltscale}{1.2}

\DeclareMathOperator{\softmax}{\text{softmax}}
\DeclareMathOperator{\sigmoid}{\text{sigmoid}}

\author{Головин Вячеслав Сергеевич}
\title{Использование обучения с подкреплением для решения задачи распознавания
диктора в интерактивном режиме}
\date{2023}

\begin{document}

\filltitle{ru}{
    chair              = {},
    title              = {Использование обучения с подкреплением для решения задачи распознавания диктора в интерактивном режиме},
    % Здесь указывается тип работы. Возможные значения:
    %   coursework - Курсовая работа
    %   diploma - Диплом специалиста
    %   master - Диплом магистра
    %   bachelor - Диплом бакалавра
    type               = {master},
    position           = {студент},
    group              = {MMO211C},
    author             = {Головин Вячеслав Сергеевич},
    supervisorPosition = {к.\,т.\,н., доцент},
    supervisor         = {Шуранов Е.\,В.},
    reviewerPosition   = {м.н.с.},
    reviewer           = {Рыжиков А.\,С.},
    chairHeadPosition  = {hello},
    chairHead          = {world},
    university = {ФЕДЕРАЛЬНОЕ ГОСУДАРСТВЕННОЕ АВТОНОМНОЕ ОБРАЗОВАТЕЛЬНОЕ
    УЧРЕЖДЕНИЕ ВЫСШЕГО ОБРАЗОВАНИЯ «НАЦИОНАЛЬНЫЙ ИССЛЕДОВАТЕЛЬСКИЙ УНИВЕРСИТЕТ
    «ВЫСШАЯ ШКОЛА ЭКОНОМИКИ»},
    faculty = {Факультет Санкт-Петербургская школа физико-математических и компьютерных наук},
    city = {Санкт-Петербург},
    year             = {2023}
}

% \maketitle

\begin{abstract}
    Мы исследовали подход к решению задачи распознавания диктора в интерактивном
    режиме. Его суть заключается в использовании обучения с подкреплением для
    создания нейросетевого агента, выбирающего запрашиваемые у диктора слова
    в зависимости от текущего контекста. Исследованный метод позволяет повысить
    точность распознавания, однако преимущество над простым эвристическим
    алгоритмом в среднем оказывается небольшим. Выполнена адаптация модели под
    практическую задачу верификации пользователя, установлено, что внесенные
    модификации не ведут к деградации результатов. Также установлено, что
    использование других эмбеддингов и обучение модели в более тяжелом режиме
    позволяет дополнительно повысить точность распознавания.
\end{abstract}

\textbf{Ключевые слова:} распознавание диктора, идентификация диктора,
верификация диктора, глубокое обучение, обучение с подкреплением.

\begin{otherlanguage}{english}
\begin{abstract}
    We study an interactive approach to speaker verification problem. Its main
    idea is the use of reinforcement learning for training a neural network,
    which selects the next requested word depending on the context. This method
    allows for increasing recognition accuracy, however on average it only
    slightly outperforms a simple heuristic algorithm. The model is adapted for
    the practical speaker verification model, we show that the modifications we
    introduct do not lead to performance degradation. We also show that the use
    of different embeddings and training the model in more challenging settings
    allows for a further increase in recognition accuracy.
\end{abstract}
\end{otherlanguage}

\textbf{Keywords:} speaker recognition, speaker identification, speaker
verification, deep learning, reinforcement learning.

\newpage
\tableofcontents

\newpage
\section*{Введение}\label{sec:intro}

Данная работа посвящена интерактивному подходу к решению задачи распознавания
диктора (\textit{Speaker Recognition})\footnote{
    Здесь и далее для ясности мы иногда будем указывать более распространённые
    названия терминов на английском языке.
}. Оригинальный метод был предложен в \citeisr, и значительная часть
работы посвящена его описанию и практической реализации. C момента публикации
этой статьи (2020~год) уже прошло достаточное количество времени, но она не
стала популярной --- по данным \textit{Google Scholar} на момент написания
этого отчёта она была процитирована 6~раз\footnote{
    Одна из этих цитат --- диссертация первого автора статьи.
}.
Тем не менее, нам (автору, его научному руководителю и коллегам из лаборатории
Huawei CBG AI) она показалась заслуживающей внимания. На это есть ряд причин.

В первую очередь стоит отметить оригинальность предложенного подхода.
Большинство работ, в той или иной степени затрагивающие задачу распознавания
диктора, посвящены способам как можно лучше определять диктора на основе уже
существующих аудиозаписей. Рассматриваемая работа ставит проблему иначе ---
какие слова или фразы должен произнести диктор, чтобы уже существующая система
смогла распознать его как можно быстрее и надёжнее. Чем-то такой подход
напоминает концепцию активного обучения (\textit{active learning}) --- разметки
только тех данных, которые являются наиболее важными для решающей функции.

Другой причиной интереса к работе стала возможность её потенциального
использования в конечном продукте --- системе аутентификации пользователя на
мобильном устройстве или персональном ассистенте. Предополагается, что такая
система будет спрашивать пользователя произнести ту или иную фразу, пока она не
станет уверена, что перед ней действительно находится настоящий владелец
прибора. В таком случае логично делать не случайные запросы, а такие, которые
позволят системе как можно быстрее идентифицировать пользователя. При этом, как
будет пояснено далее, возможность делать разнообразные запросы тоже является
преимуществом. Кроме того, этот подход может быть использован и для других
задач, например, для синтеза речи с определённым голосом.

Таким образом, целью данной работы является разработка системы распознавания
диктора, в которой высокая точность достигается за счёт выбора запрашиваемых
у диктора слов. Первоочерёдной задачей работы стало воспроизведение результатов,
достигнутых в \citeisr{}. Другими задачами стали адаптация изначальной модели
для целевого конечного продукта (системы аутентификации пользователя) и
повышение её точности.

Сформулированные задачи определяют структуру отчёта. В главе~\ref{sec:theory}
дано подробное описание исследуемого метода. Глава~\ref{sec:experiments}
посвящена вопросам имплементации и полученным при этом результатам. Модификации
изначальной системы (например, переход к задаче верификации) рассматриваются в
главе~\ref{sec:modifications}.

\newpage
\section{Распознавание диктора в интерактивном режиме}\label{sec:theory}

\subsection{Задача распознавание диктора}
    \ldots

\subsection{Интерактивный режим}\label{ssec:isr}
    \ldots

\newpage
\section{Детали реализации и результаты}\label{sec:experiments}

\subsection{Данные для обучения и извлечение эмбеддингов}\label{ssec:data}

Здесь мы практически полностью повторяем описанный в~\cite{isrpaper} подход.
Единственным (но очень существенным) отличием является использованная размерность
эмбеддингов. Перед тем как перейти к обсуждению этого момента, расскажем про
исходные данные.

Итак, для обучения и тестирования моделей мы использовали датасет
TIMIT\cite{timit}. Он составлен из аудиозаписей речи 630~дикторов, говорящих на
8~основных диалектах американского английского языка. Эти дикторы поделены на
обучающую (\texttt{train}) и тестовую (\texttt{test}) выборки, в первую входят
468~дикторов, во вторую --- 162. Для обучения нейросетевых моделей мы также
создавали валидационную выборку, в которую выделялись 20\% дикторов из обучающей.

Каждый из дикторов произносит 10~фонетически насыщенных предложений. При этом 2
из 10 предложений являются общими для всех дикторов
\footnote{Общие предложения:\\
          \textit{She had your dark suit in greasy wash water all year.}\\
          \textit{Don't ask me to carry an oily rag like that.}}
, остальные 8 уникальны для каждого диктора. Такое разделение позволяет без
особых затруднений подготовить данные, необходимые для описанной
в~\ref{ssec:isr} игры:
\begin{itemize}
    \item 2~общих предложения можно использовать для получения аудиозаписей
    слов.  Для этого разделим аудиозаписи этих предложений по временным
    отметкам, предоставленным создателями датасета. В результате получим 20
    аудиозаписей слов\footnote{Аналогично~\cite{isrpaper} мы не используем слово
    \textit{an}.} для каждого диктора.
    \item 8~уникальных для каждого диктора предложений можно использовать для
    получения голосовых подписей --- эмбеддингов дикторов --- просто при помощи
    усреднения эмбеддингов аудиозаписей этих предложений.
\end{itemize}

В качестве векторов признаков использовались эмбеддинги
\xvector{}~\cite{xvectorspaper}. Весь процесс преобразования аудиозаписей в
векторы признаков осуществлялся с помощью библиотеки Kaldi~\cite{kaldi}. На
первом этапе рассчитывалиcь мел-частотные кепстральные коэффициенты\footnote{
Параметры аналогичны использованным в~\cite{isrpaper} и определяются
требованиями предобученной модели.} и производилось детектирование голосовой
активности (\textit{aнгл.} VAD --- voice activity detection). Полученные векторы
признаков поступали на вход предобученной нейронной сети~\cite{sre16model}. В
качестве эмбеддингов использовались данные со второго 512-мерного слоя.

Здесь, как уже было сказано ранее, мы отступаем от оригинальной
работы~\cite{isrpaper}, где использовались 128-мерные эмбеддинги. На это есть
две причины. Во-первых, из приведенных в~\cite{isrpaper} комментариев
неочевидно\footnote{
    Цитата: \textit{We then process the MFCCs features through a pretrained
    X-Vector network to obtain a high quality voice embedding of fixed dimension
    128, where the X-Vector network is trained on augmented Switchboard, Mixer~6
    and NIST SREs}.
}, как производилось понижение размерности.
Во-вторых, мотивация такого преобразования тоже неочевидна. Уже первые проведенные
нами эксперименты показали, что при использовании 512-мерных эмбеддингов точность
идентификации оказывается существенно выше приведенных в~\cite{isrpaper} значений.

\subsection{Обучение \guesser{}}

Первой обучается нейронная сеть \guesser{}, выполняющая выбор из $K$ дикторов
при помощи $T$ аудиозаписей произнесенных слов. Как уже было сказано ранее,
эта нейросеть тренируется в режиме обучения с учителем, дикторы и произносимые
слова выбираются случайно, в качестве функции используется кросс-энтропия.
Процесс вычисления значения функции потерь для одной игры можно записать
следующим образом:
\begin{lstlisting}[caption={Рассчёт функции потерь \guesser{}}]
speaker_ids = speakers.sample(size=K)
G = voice_prints.get(speaker_ids)
target = randrange(0, K)
word_inds = randrange(0, V, size=T)
X = word_vocab.get(speaker=speaker_ids[target],
                   words=word_inds)
probabilities = guesser.forward(G, X)
loss = cross_entropy(probabilities, target)
\end{lstlisting}

Из-за увеличения относительно \citeisr{} размерности эмбеддингов пропорционально
увеличились и размерности слоёв \guesser{}. Из-за этого нам пришлось изменить
гиперпараметры, в частности мы сильно уменьшили темп обучения
(\textit{learning rate}).

Как и в оригинальной статье, для сравнения моделей будем строить графики
\textit{word} и \textit{guest sweep}. Т.~е. будем обучать модель в режиме с $K =
5$ дикторами и $T = 3$ запрашиваемыми словами, а затем будем тестировать её в
режимах с отличным числом дикторов или слов. Здесь и далее, если это не оговорено
отдельно, для расчёта точности проводятся $20000$ игр среди дикторов из тестовой
выборки, эксперименты повторяются по 5~раз c различным \texttt{seed} генератора
случайных чисел.

\begin{figure}[!h]
    \centering
    \includegraphics[scale=1.0]{../plots/guest_sweep.pdf}
    \caption{Зависимость точности \guesser{} обученного нами и авторами
    \citeisr{} от числа дикторов $K$. Модели обучены в режиме $K = 5$, $T = 3$.}
\end{figure}

\begin{figure}[!h]
    \centering
    \includegraphics[scale=1.0]{../plots/word_sweep.pdf}
    \caption{Зависимость точности \guesser{} обученного нами и авторами
    \citeisr{} от числа запрошенных слов $T$. Модели обучены в режиме $K = 5$,
    $T = 3$.}
\end{figure}

По приведенным на графиках результатам видно, что увеличение размерности
эмбеддингов существенно улучшает точность идентификации, разница особо
велика в режимах с большим числом дикторов $K$.

\subsection{Обучение \enquirer{}}

Для обучения \enquirer{} --- модели выбора слов --- уже нужна обученная модель
\guesser{}. На этом этапе уже используется обучение с подкреплением, псевдокод
для 1~игры приведён ниже.
\begin{verbatim}
speaker_ids = speakers.sample(size=K)
G = voice_prints.get(speaker_ids)
target = randrange(0, K)

g_hat = G.mean(dim=0)
x_i = start_tensor
X = []
for i in range(T):
    probs = enquirer.forward(g_hat, x_i)
    if training:
        word_inds = multinomial(probs).sample()
    else:
        word_ind = argmax(probs)
    x_i = word_vocab.get(speaker=speaker_ids[target], word=word_ind)
    X.append(x_i)

prediction = guesser.predict(G, X)
reward = 1 if prediction == target else 0
\end{verbatim}

Как видно из приведенного псевдокода, награда выдается в том случае, когда
\guesser{} правильно угадывает диктора. Для обучения мы использовали алгоритм
PPO~\cite{schulman2017proximal} --- здесь мы снова повторяем подход
авторов~\citeisr{}. В целом выбор метода выглядит разумным --- PPO
зарекомендовал себя как простой и универсальный алгоритм, позволяющий достигать
хороших результатов. Однако некоторые особенности нашей задачи --- дискретное
пространство действий, малая длительность эпизодов --- выглядят лучше подходящими
для off-policy алгоритмов. К сожалению, у нас не нашлось времени, чтобы проверить
эту гипотезу.

\begin{figure}[!h]
    \centering
    \includegraphics[scale=1.0]{../plots/guest_sweep_enq.pdf}
    \caption{Зависимость точности SR-систем с различными методами выбора слов
    --- нейросетевым агентов (\enquirer{}) и случайным (random agent) --- от
    числа дикторов $K$. Модели обучены в режиме $K = 5$, $T = 3$.}
\end{figure}

\begin{figure}[!h]
    \centering
    \includegraphics[scale=1.0]{../plots/word_sweep_enq.pdf}
    \caption{Зависимость точности SR-систем с различными методами выбора слов
    --- нейросетевым агентов (\enquirer{}) и случайным (random agent) --- от
    числа запрашиваемых слов $T$. Модели обучены в режиме $K = 5$, $T = 3$.}
\end{figure}

Приведенные результаты свидетельствуют о том, что \enquirer{} действительно
успешно обучается --- точность оказываются заметно выше, чем с случае
случайного выбора слов. Как и в случае повышения размерности эмбеддингов,
особенно большое различие наблюдается в режимах с большим числом дикторов.

\subsection{Эвристическая модель выбора слов}

Очевидно, что агент, выбирающий запрашиваемые слова случайным образом, не
является тяжелым противником для нейросетевого агента. Для более трезвой
оценки возможностей последнего, логично сравнивать его с каким-то более сложным
алгоритмом.

Здесь мы снова немного отходим от оригинальной статьи. И опять основной
причиной является тот факт, что в \citeisr{} отсутствует точное описание
использованного в качестве бейзлайна эвристического алгоритма выбора слов.
Из приведенных в работе слов\footnote{
    Цитата: \textit{We curated a list of the most discriminant
    words (words that increase globally the recognition scores)
    and sample among those instead of the whole list.}
}
общий подход понятен --- сэмплирование производится не из всех $20$ слов, а
из тех, которые в среднем показывают самую высокую точность. При этом остаются
непонятными следующие детали:
\begin{enumerate}
    \item Из скольки слов производится сэмплирование, и меняется ли это число
    в зависимости от числа запрашиваемых слов $T$?
    \item Производится ли сэмплирование равномерно, или вероятность выбрать
    слово пропорциональна достигаемой при выборе этого слова средней точности?
\end{enumerate}

Именно такие вопросы возникли у нас при создании эвристического агента. Первым
же этапом стала оценка слов --- расчёт средней точности, которая достигается
случайным агентом в тех играх, когда он выбрал то или иное слово. Для этого мы
протестировали \guesser{} в $100000$ эпизодов с $K = 5$, и $T = 3$, а также
случайным выбором слов без повторений. Мы рассчитывали точность для каждого слова,
учитывая только те эпизоды, в которых это слово было выбрано. Фактически мы оценивали
условную вероятность связки \guesser{}--случайный агент правильно выбрать диктора
при условии, что одно слово уже было выбрано. \hl{ссылка на рисунок}

\begin{figure}[!h]
    \centering
    \includegraphics[scale=1.0]{../plots/word_scores.pdf}
    \caption{Средняя точность \guesser{} на валидационной выборке в тех
    эпизодах, когда соответствующее слово было выбрано (остальные выбирались
    случайно).}
\end{figure}

После этого мы стали тестировать различные модификации эвристического агента.
Как следует из сформулированных выше вопросов, эти агенты отличались числом
использованных слов и методом сэмплирования. Эксперименты показали, что
наилучшие результаты достигаются при использовании ``детерминированного'' агента,
всегда выбирающего одни и те же слова с наибольшей средней точностью.
В таком случае говорить о каком-либо сэмплировании неуместно, поэтому такой
агент, по всей видимости, отличается от использованного в оригинальной статье.

\begin{figure}[!h]
    \centering
    \includegraphics[scale=1.0]{../plots/guest_sweep_heuristic.pdf}
    \caption{Зависимость точности SR-систем с различными методами выбора слов
    --- нейросетевым и эвристическим агентами --- от
    числа дикторов $K$. Модели обучены в режиме $K = 5$, $T = 3$.}
\end{figure}

\begin{figure}[!h]
    \centering
    \includegraphics[scale=1.0]{../plots/word_sweep_heuristic.pdf}
    \caption{Зависимость точности SR-систем с различными методами выбора слов
    --- нейросетевым и эвристическим агентами --- от
    числа запрашиваемых слов $T$. Модели обучены в режиме $K = 5$, $T = 3$.}
\end{figure}

В данном случае преимущество \enquirer{} проявляется только в режимах с большим
числом дикторов. В стандартном режиме с $K = 5$ дикторами и $T = 3$, в отличие
от~\citeisr{}, мы не наблюдаем сколько-нибудь существенной разницы между двумя
агентами.

В таком случае возникает резонный вопрос --- не сходится ли \enquirer{}
к такой же политике, что использует эвристический агент? Ответ на этот вопрос ---
отрицательный, в целом \enquirer{} выбирает из 5 слов (ещё 2 используются редко),
в то время как эвристический агент всегда использует 3 тех же слова. Из этого
можно предположить, что \enquirer{} обучен недостаточно хорошо, возможно, другие
гиперпараметры или алгоритм обучения позволили бы улучшить результаты.

\subsection{Обучение в других режимах}

Другой логичный вопрос, возникающий при обсуждении графиков \textit{word} и
\text{guest sweep} --- является ли стандартный режим ($K = 5$ дикторов и
$T = 3$ запрашиваемых слова) оптимальным для обучения моделей? Не будут ли
результаты лучше, если мы будем обучать и тестировать модели в одном и том же
режиме? Мы также провели ряд экспериментов и пришли к следующим выводам:
\begin{itemize}
    \item Общее правило --- более тяжелые режимы позволяют улучшить точность.
    В первую очередь это касается увеличения числа дикторов, ситуация с
    уменьшением числа слов менее однозначная.
    \item Основные улучшения наблюдаются в работе \guesser{}, в то же время
    \enquirer{} оказывается нечувствительным к режиму обучения.
    \item Обучение при $T = 1$ является специфической задачей. Во-первых,
    только в этом режиме обученная модель показывает хорошие (лучше чем другие)
    результаты только в этом же режиме тестирования. Во-вторых, \enquirer{}
    в среднем показывает плохие результаты в данном режиме, часто уступая
    максимально простой политике, всегда выбирающей одно и то же слово.
\end{itemize}

\newpage
\section{Модификации метода}\label{sec:modifications}

\subsection{От идентификации к верификации}
\enquirer{} менять вообще не нужно, \guesser{} --- совсем немного.

\subsection{\cbenquirer}
вроде работает

\subsection{Добавление шума}
обучается норм, результаты такие же

\subsection{Альтернативные эмбеддинги}\label{ssec:cpc}
внезапно эмбеддинги из 2018 года оказались не очень


\newpage
\section*{Заключение}

все работает, но хотелось бы большего

\bibliographystyle{ieeetr}
\bibliography{references.bib}
\end{document}