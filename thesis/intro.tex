Данная работа посвящена интерактивному подходу к решению задачи распознавания
диктора.  Оригинальный метод был предложен в \citeisr, и значительная часть
работы посвящена его описанию и практической реализации. C момента публикации
этой статьи (2020~год) уже прошло достаточное количество времени, но она не
стала популярной --- по данным \textit{Google Scholar} на момент написания
этого отчёта она была процитирована 6~раз\footnote{
    Из этих цитат 1 приходится на кандидатскую диссертацию её первого автора.
}.
Тем не менее, нам (автору дипломной работы, его научному руководителю и коллегам
из лаборатории Huawei CBG AI) она показалась заслуживающей внимания. На это есть
ряд причин.

В первую очередь стоит отметить оригинальность предложенного подхода.
Исторически большинство работ, в той или иной степени затрагивающие задачу
распознавания диктора, посвящены способам как можно лучше определять диктора
на основе уже существующих аудиозаписей. \hl{здесь, наверное, нужно привести
примеры таких работ} Рассматриваемая работа ставит проблему иначе --- какие
слова или фразы должен произнести диктор, чтобы уже существующая система
смогла распознать его как можно быстрее и надёжнее. Чем-то такой подход
напоминает концепцию активного обучения (\textit{англ.} active learning) ---
разметки только тех данных, которые являются наиболее важными для решающей
функции.

Другой причиной интереса к работе стала возможность её потенциального использования
в конечном продукте --- системе аутентификации пользователя на мобильном устройстве
или персональном ассистенте. Предополагается, что такая система будет спрашивать
пользователя произнести ту или иную фразу, пока она не станет уверена, что перед
ней действительно находится настоящий владелец прибора. В таком случае логично
делать не случайные запросы, а такие, которые позволят системе как можно быстрее
идентифицировать пользователя. 

Более подробное описание метода дано в главе~\ref{sec:theory}. Следующая глава
посвящена практической реализации описанного метода и полученным результатам.
Глава~\ref{sec:modifications} в свою очередь посвящена модификациям
оригинального подхода, направленными на повышение точности и адаптации метода
под сформулированную выше практическую задачу.