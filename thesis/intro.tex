Данная работа посвящена интерактивному подходу к решению задачи распознавания
диктора (\textit{speaker recognition}, \textit{SR})\footnote{
    Здесь и далее для ясности мы иногда будем указывать более распространённые
    названия терминов на английском языке.
}. Оригинальный метод был предложен в \citeisr, и значительная часть
работы посвящена его описанию и практической реализации. C момента публикации
этой статьи (2020~год) уже прошло достаточное количество времени, но она не
стала популярной --- по данным \textit{Google Scholar} на момент написания
этого отчёта она была процитирована 6~раз\footnote{
    Из этих цитат 1 приходится на диссертацию её первого автора.
}.
Тем не менее, нам (автору, его научному руководителю и коллегам из лаборатории
Huawei CBG AI) она показалась заслуживающей внимания. На это есть ряд причин.

В первую очередь стоит отметить оригинальность предложенного подхода.
Большинство работ, в той или иной степени затрагивающие задачу распознавания
диктора, посвящены способам как можно лучше определять диктора на основе уже
существующих аудиозаписей. Рассматриваемая работа ставит проблему иначе ---
какие слова или фразы должен произнести диктор, чтобы уже существующая система
смогла распознать его как можно быстрее и надёжнее. Чем-то такой подход
напоминает концепцию активного обучения (\textit{active learning}) --- разметки
только тех данных, которые являются наиболее важными для решающей функции.

Другой причиной интереса к работе стала возможность её потенциального
использования в конечном продукте --- системе аутентификации пользователя на
мобильном устройстве или персональном ассистенте. Предополагается, что такая
система будет спрашивать пользователя произнести ту или иную фразу, пока она не
станет уверена, что перед ней действительно находится настоящий владелец
прибора. В таком случае логично делать не случайные запросы, а такие, которые
позволят системе как можно быстрее идентифицировать пользователя. При этом, как
будет пояснено далее, возможность делать разнообразные запросы тоже является
преимуществом. Кроме того, этот подход может быть использован и для других
задач, например, для синтеза речи с определённым голосом.

Более подробное описание метода дано в главе~\ref{sec:theory}.
Глава~\ref{sec:experiments} посвящена вопросам практической реализации и
полученным результатам. В главе~\ref{sec:modifications} представлены модификации
оригинального подхода, направленные на повышение точности и адаптацию метода под
сформулированную выше прикладную задачу.