Главным вывод проделанной работы --- исследованный метод имеет право на
существование, он действительно позволяет повысить точность распознавания
диктора, поэтому его можно встраивать в ряд существующих систем. При этом
его эффективность оказалась не самой впечатляющей --- во многих режимах он
показал сравнимые результаты с очень простым эвристическим агентом. Возможно,
проблема заключается в неоптимальном режиме обучения. В частности, имеет смысл
попробовать offline алгоритмы обучения с подкреплением.

Другим направлением для исследований может являться архитектура использованных
нейросетей. В частности, текущая имплементация \guesser{} никак не использует
информацию о том, какое слово было произнесено, а \enquirer{} не знает о
текущем состоянии \guesser{}. Также, наверное, имеет смысл попробовать более
сложные архитектуры --- вполне возможно, что они приведут к росту точности,
сравнимому с тем, что был достигнут за счёт использования более современных
эмбеддингов.